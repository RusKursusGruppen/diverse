\documentclass[10pt,a4paper,letterpaper]{article}

% Usual packages++
\usepackage[utf8]{inputenc}
\usepackage[T1]{fontenc}
\usepackage[danish]{babel}
\usepackage{amsfonts}
\usepackage[hidelinks]{hyperref}
\usepackage[usenames, dvipsnames, svgnames, table]{xcolor}
\usepackage[yyyymmdd]{datetime}
\usepackage[a4paper, total={6in, 8in}]{geometry}

% Basic layout:
\setlength{\textwidth}{165mm}
\setlength{\textheight}{220mm}
\setlength{\parindent}{0mm}
\setlength{\parskip}{\parsep}
\setlength{\headheight}{0mm}
\setlength{\headsep}{0mm}
\setlength{\hoffset}{-2.5mm}
\setlength{\voffset}{0mm}
\setlength{\footskip}{15mm}
\setlength{\oddsidemargin}{0mm}
\setlength{\topmargin}{0mm}
\setlength{\evensidemargin}{0mm}

\newcolumntype{L}[1]{>{\raggedright\let\newline\\\arraybackslash\hspace{0pt}}m{#1}}
\newcolumntype{C}[1]{>{\centering\let\newline\\\arraybackslash\hspace{0pt}}m{#1}}

\begin{document}

\thispagestyle{empty}
{\large{\textbf{Praktisk info: Pusleseddel \the\year}}}\\
Dette er puslesedlen til datalogiruskursushytteholdene \the\year. I år pusler vi $n$
hold. Herunder følger en dværgpædagogisk instruktion i at udfylde puslesedlen.
Den er dværgpædagogisk, fordi folk traditionelt ikke kan finde ud af at gøre det
ordentligt, så læs hvad der står!
Puslesedlen består af to sider

\begin{itemize}
  \item \textbf{Side 1: Informationer om dig selv} \\
    Alle felter skal udfyldes.
  \item \textbf{Side 2: Puslesedlen} \\
Du bliver puslet ved at udfylde denne seddel. Du skal kun udfylde puslesedlen,
hvis du vil lave rustur i år. Sørg for at udfylde sedlen så fyldestgørende og
korrekt som muligt - det er i din egen interesse.

Ud for hver person på navnelisten er der syv kasser.
\begin{itemize}
  \item \textbf{Vil ikke}: Benyt kun denne kasse hvis du absolut ikke vil lave rustur hvis det er sammen med vedkomne
  \item \textbf{Ønsker ikke}: Benyt denne kasse hvis du helst ikke ønsker at lave rustur med vedkomne.
  \item \textbf{Måske Neg}: Benyt denne kasse hvis du har nogle forbehold mod at lave rustur med vedkomne.
  \item \textbf{Måske Pos}: Benyt denne kasse hvis du ikke vil have noget imod at lave rustur med vedkomne.
  \item \textbf{Vil gerne}: Benyt denne kasse hvis du gerne vil lave rustur med vedkomne.
  \item \textbf{Vil}: Benyt denne kasse hvis du meget gerne vil lave rustur med vedkomne.
  \item \textbf{Var vejleder med sidste år}: Benyt denne kasse til at nævne hvem du
    lavede rustur med sidste år, hvis nogen.
\end{itemize}
Hvis du er neutral, så er der ikke behov for at sætte nogen krydser.

Der skal ikke være noget kryds ud for dig selv, men du skal sætte en streg under
dit navn.\\
\end{itemize}

{\large{\textbf{Hvornår bliver der puslet?}}}\\
\begin{itemize}
    \item Puslingen foregår Onsdag d. 10 juli \the\year, indtil der på et
      tidspunkt er enighed om holdende.
    \item Derudover anbefales det, at du tager et billede/beholder en identisk
      version af din pusleseddel, så du ikke glemmer, hvordan de pusling nu
      lige så ud.
    \item Hvis du placerer mange personer i ”Ønsker ikke” kolonnen må du være
      indstillet på, at du kommer på hold med flere af disse personer. Jo færre
      krydser i kolonnen, jo mere sandsynligt er det, at dine ønsker kan
      tilgodeses.
    \item Puslesedlerne vil naturligvis blive behandlet strengt fortroligt,
      makuleret, brændt, ødelagt og destrueret, og vi vil udradere vores
      korttidshukommelse med sprut i store mængder.
    \item Vi ringer rundt på aftenen, så skriv et telefonnummer, hvor vi kan få
      fat på dig mellem kl 19.00 og 03.00-ish, og muligvis også senere. Hvis vi
      ikke kan få fat på dig inden for et rimeligt tidsrum, betragtes det, som
      om, at du har godkendt dit hold pr.\ default.
    \item Skal du ikke med på 2. nar, og er du evt.\ i udlandet, bedes du udtænke
      en måde, hvorpå du alligevel kan kontaktes og notere dette på din
      pusleseddel, eventuelt i form af en fuldmagt.
\end{itemize}

\newpage
\setcounter{page}{1}
{\large{\textbf{Pusleseddel \the\year}}}\\

\makebox[\textwidth]{Navn:\enspace\hrulefill}\\

\makebox[\textwidth]{Adresse:\enspace\hrulefill}\\

\makebox[0.5\textwidth]{Postnr./by:\enspace\hrulefill}
\makebox[0.5\textwidth]{Tlf.\ nr.:\enspace\hrulefill} \\

\begin{tabular}{L{6cm}rrr}
Jeg har kørekort: &
Ja $\square$ & Nej $\square$ & Ja, men vil helst ikke køre $\square$ \\

Jeg kan lægge hus ud til 3. nar: &
Ja $\square$ & Nej $\square$ & Muligvis $\square$ \\

Jeg ønsker at komme på: &
Pigehold $\square$ & Neutral $\square$ & Munkehold $\square$ \\

Til september \the\year\ læser jeg på: &
& Bachelor $\square$ & Kandidat $\square$ \\

Jeg har re-eksmanen ved sommerstart: &
Ja $\square$ & Nej $\square$ & Muligvis $\square$ \\

Jeg deltog til svedtur(?): &
Ja, i år \rule{1cm}{0.3pt} $\square$ & Nej $\square$ & \\
\end{tabular} \\
\vspace*{0.5cm}

Jeg har læst på DIKU i \rule{1cm}{0.3pt} år\\

Jeg bor omkring \rule{1cm}{0.3pt} km fra DIKU
\\

\vspace*{1cm}
Evt.\ planlagt ferie:
\enspace\hrulefill\null\hrulefill{}

\null\hrulefill{}

\null\hrulefill{} \\

\vspace*{1cm}
Hvad laver jeg efter rusturen og resten af studieåret?
\enspace\hrulefill\null\hrulefill{}

\null\hrulefill{}

\null\hrulefill{} \\

\vspace*{1cm}
Evt./Bemærkninger:
\enspace\hrulefill\null\hrulefill{}

\null\hrulefill{}

\null\hrulefill{} \\

\newpage
Jeg har følgende ønsker til puslingen (Sæt \'et kryds pr.\ person)\\

\begin{tabular}{L{3.3cm}C{1.5cm}C{1.5cm}C{1.5cm}C{1.5cm}C{1.5cm}C{1.5cm}C{1.8cm}}
  \textbf{Sæt streg under dit eget navn} &
  \textbf{Vil}                           &
  \textbf{Ønsker gerne}                  &
  \textbf{Positiv}                       &
  \textbf{Negativ} &
  \textbf{Ønsker ikke} &
  \textbf{Vil ikke} &
  \textbf{Har været der} \\

Navn & $\square$ & $\square$ & $\square$ & $\square$ & $\square$ & $\square$ & $\square$ \\
Navn & $\square$ & $\square$ & $\square$ & $\square$ & $\square$ & $\square$ & $\square$ \\
Navn & $\square$ & $\square$ & $\square$ & $\square$ & $\square$ & $\square$ & $\square$ \\
Navn & $\square$ & $\square$ & $\square$ & $\square$ & $\square$ & $\square$ & $\square$ \\
Navn & $\square$ & $\square$ & $\square$ & $\square$ & $\square$ & $\square$ & $\square$ \\
Navn & $\square$ & $\square$ & $\square$ & $\square$ & $\square$ & $\square$ & $\square$ \\
Navn & $\square$ & $\square$ & $\square$ & $\square$ & $\square$ & $\square$ & $\square$ \\
Navn & $\square$ & $\square$ & $\square$ & $\square$ & $\square$ & $\square$ & $\square$ \\
Navn & $\square$ & $\square$ & $\square$ & $\square$ & $\square$ & $\square$ & $\square$ \\
Navn & $\square$ & $\square$ & $\square$ & $\square$ & $\square$ & $\square$ & $\square$ \\
Navn & $\square$ & $\square$ & $\square$ & $\square$ & $\square$ & $\square$ & $\square$ \\
Navn & $\square$ & $\square$ & $\square$ & $\square$ & $\square$ & $\square$ & $\square$ \\
Navn & $\square$ & $\square$ & $\square$ & $\square$ & $\square$ & $\square$ & $\square$ \\
Navn & $\square$ & $\square$ & $\square$ & $\square$ & $\square$ & $\square$ & $\square$ \\
Navn & $\square$ & $\square$ & $\square$ & $\square$ & $\square$ & $\square$ & $\square$ \\
Navn & $\square$ & $\square$ & $\square$ & $\square$ & $\square$ & $\square$ & $\square$ \\
Navn & $\square$ & $\square$ & $\square$ & $\square$ & $\square$ & $\square$ & $\square$ \\
Navn & $\square$ & $\square$ & $\square$ & $\square$ & $\square$ & $\square$ & $\square$ \\
Navn & $\square$ & $\square$ & $\square$ & $\square$ & $\square$ & $\square$ & $\square$ \\
Navn & $\square$ & $\square$ & $\square$ & $\square$ & $\square$ & $\square$ & $\square$ \\
Navn & $\square$ & $\square$ & $\square$ & $\square$ & $\square$ & $\square$ & $\square$ \\
Navn & $\square$ & $\square$ & $\square$ & $\square$ & $\square$ & $\square$ & $\square$ \\
Navn & $\square$ & $\square$ & $\square$ & $\square$ & $\square$ & $\square$ & $\square$ \\
Navn & $\square$ & $\square$ & $\square$ & $\square$ & $\square$ & $\square$ & $\square$ \\
Navn & $\square$ & $\square$ & $\square$ & $\square$ & $\square$ & $\square$ & $\square$ \\
Navn & $\square$ & $\square$ & $\square$ & $\square$ & $\square$ & $\square$ & $\square$ \\
Navn & $\square$ & $\square$ & $\square$ & $\square$ & $\square$ & $\square$ & $\square$ \\
Navn & $\square$ & $\square$ & $\square$ & $\square$ & $\square$ & $\square$ & $\square$ \\
Navn & $\square$ & $\square$ & $\square$ & $\square$ & $\square$ & $\square$ & $\square$ \\
Navn & $\square$ & $\square$ & $\square$ & $\square$ & $\square$ & $\square$ & $\square$ \\
Navn & $\square$ & $\square$ & $\square$ & $\square$ & $\square$ & $\square$ & $\square$ \\
Navn & $\square$ & $\square$ & $\square$ & $\square$ & $\square$ & $\square$ & $\square$ \\
Navn & $\square$ & $\square$ & $\square$ & $\square$ & $\square$ & $\square$ & $\square$ \\
Navn & $\square$ & $\square$ & $\square$ & $\square$ & $\square$ & $\square$ & $\square$ \\
Navn & $\square$ & $\square$ & $\square$ & $\square$ & $\square$ & $\square$ & $\square$ \\
Navn & $\square$ & $\square$ & $\square$ & $\square$ & $\square$ & $\square$ & $\square$ \\
Navn & $\square$ & $\square$ & $\square$ & $\square$ & $\square$ & $\square$ & $\square$
\end{tabular}

\vspace*{1cm}
\textit{Har været der = Har været vejleder med før}

\end{document}
